\documentclass[12pt]{article}
\usepackage[utf8]{inputenc} % input encoding
\usepackage[T1]{fontenc} % Use an 8-bit font encoding, so that ã is a
                         % single glyph in the font. Yields better
                         % hyphenation and better cut-and-paste from pdf
\usepackage{times} % Comment this line you you want the default font, Computer Roman
% \usepackage[portuguese]{babel} % Uncomment this file you plan to write in Portuguese
\usepackage{hyperref}
\usepackage{listings}
\usepackage{graphicx}
\usepackage[section]{placeins}
\usepackage{multicol}
\usepackage{tabularx}
\usepackage{makecell}
\graphicspath{ {./images/} }
\sloppy

\lstset{frame=tb,
  aboveskip=3mm,
  belowskip=3mm,
  showstringspaces=false,
  columns=flexible,
  basicstyle={\small\ttfamily},
  numbers=none,
  breaklines=true,
  breakatwhitespace=true,
  tabsize=3
}

\title{TST JUnit Testing \\
  \Large Software Verification and Validation \\ 2019--2020
}
\author{
  João David\\49448
  \and
  Ye Yang\\49521
}
\date{09/05/2020}

\begin{document}
\maketitle

\section{Instruction Coverage}
\subsection{size()}
\begin{lstlisting}
    public int size() {
        return n; //I1
    }
\end{lstlisting}

\begin{table}[htb]
\centering
\begin{tabular}{| c | c | c | c |} 
 \hline
 Test Case & Values & Expected / Actual & IC\\ \hline
 sizeZeroTest & - & 0 & I1 \\ \hline

\end{tabular}
\end{table}

\subsection{contains(String key)}
\begin{lstlisting}
    public boolean contains(String key) {
        if (key == null) 
            throw new IllegalArgumentException("argument to contains() is null");
        return get(key) != null;
    }
\end{lstlisting}

\begin{table}[htb]
\centering
\begin{tabular}{| c | c | c | c |} 
 \hline
 Test Case & Values & Expected / Actual & IC\\ \hline
 containsNullKey & null & IAE & I1, I2 \\ \hline
 containsNonNullKey & "someKey" & false & I1, I3 \\ \hline
\end{tabular}
\end{table}

\subsection{get(String key)}
\begin{lstlisting}
public T get(String key) {
   if (key == null)
       throw new IllegalArgumentException("calls get() with null argument");
   if (key.length() == 0) 
       throw new IllegalArgumentException("key must have length >= 1");
   Node<T> x = get(root, key, 0);
   if (x == null) 
      return null;
   return x.val;
}
\end{lstlisting}

\begin{table}[htb]
\centering
\begin{tabular}{| c | c | c | c |} 
 \hline
 Test Case & Values & Expected / Actual & IC\\ \hline
 getNullKey & null & IAE & I1, I2 \\ \hline
 getEmptyStringKey & "" & IAE & I1, I3, I4 \\ \hline
 getNonExistentKey & “someKey” & null & I1, I3, I5, I6, I7 \\ \hline
 getExistentKey & “key” & <value> & I1, I3, I5, I6, I8 \\ \hline 
\end{tabular}
\end{table}

\subsection{put(String key, T val)}
\begin{lstlisting}
public void put(String key, T val) {
   if (key == null)
      throw new IllegalArgumentException("calls put() with null key");
   if (!contains(key)) 
      n++;
   root = put(root, key, val, 0);
}
\end{lstlisting}

\begin{table}[htb]
\centering
\begin{tabular}{| c | c | c | c |} 
 \hline
 Test Case & Values & Expected / Actual & IC\\ \hline
 putNullKey & null, 1 & IAE & I1, I2 \\ \hline
 putValidNewKey & “someKey”, 1 & NoExep & I1, I3, I4, I5 \\ \hline
\end{tabular}
\end{table}


\subsection{longestPrefixOf(String query)}
\begin{lstlisting}
public String longestPrefixOf(String query) {
   if (query == null)
      throw new IllegalArgumentException("calls longestPrefixOf() with null argument");
   if (query.length() == 0) 
      return null;
   int length = 0;
   Node<T> x = root;
   int i = 0;
   while (x != null && i < query.length()) {
      char c = query.charAt(i);
      if      (c < x.c) x = x.left;
      else if (c > x.c) x = x.right;
      else {
         i++;
         if (x.val != null) 
            length = i;
         x = x.mid;
      }
   }
   return query.substring(0, length);
}
\end{lstlisting}

\begin{table}[htb]
\centering
\begin{tabular}{| c | c | c | c |} 
 \hline
 Test Case & Values & Expected / Actual & IC\\ \hline
 longestPrefixOfNull & null & IAE & I1, I2 \\ \hline
 longestPrefixOfEmptyString & "" & null & I1, I3, I4 \\ \hline
 longestPrefixOfAllInstructions & "c" & "c" & \makecell{I1, I3, I5, I6,\\ I7, I8, I9, I10, \\ I11, I12, I13, I14, \\ I15, I16, I17, I18, I19} \\ \hline
\end{tabular}
\end{table}


\subsection{keys()}
\begin{lstlisting}
public Iterable<String> keys() {
   Queue<String> queue = new LinkedList<>();
   collect(root, new StringBuilder(), queue);
   return queue;
}
\end{lstlisting}

\begin{table}[htb]
\centering
\begin{tabular}{| c | c | c | c |} 
 \hline
 Test Case & Values & Expected / Actual & IC\\ \hline
 keysTest & - & Empty Iterator & I1, I2, I3 \\ \hline
\end{tabular}
\end{table}


\subsection{keysWithPrefix(String prefix)}
\begin{lstlisting}
public Iterable<String> keysWithPrefix(String prefix) {
    if (prefix == null)
        throw new IllegalArgumentException("calls keysWithPrefix() with null argument");
    Queue<String> queue = new LinkedList<>();
    Node<T> x = get(root, prefix, 0);
    if (x == null) 
        return queue;
    if (x.val != null) 
        queue.add(prefix);
        collect(x.mid, new StringBuilder(prefix), queue);
        return queue;
}
\end{lstlisting}

\begin{table}[htb]
\centering
\begin{tabular}{| c | c | c | c |} 
 \hline
 Test Case & Values & Expected / Actual & IC\\ \hline
 keysWithPrefixNull & null & IAE & I1, I2 \\ \hline
 \makecell{keysWithPrefix\\NonExistentPrefix} & "prefix" & Iterator (size 0) & I1, I3, I4, I5, I6 \\ \hline
 \makecell{keysWithPrefix\\ExistentPrefix} & "c" & Iterator (size 1) & \makecell{I1, I3, I4, I5, \\ I6, I7, I8, I9, I10} \\ \hline
\end{tabular}
\end{table}


\subsection{keysThatMatch(String pattern)}
\begin{lstlisting}
public Iterable<String> keysThatMatch(String pattern) {
    Queue<String> queue = new LinkedList<>();
    collect(root, new StringBuilder(), 0, pattern, queue);
    return queue;
}
\end{lstlisting}

\begin{table}[htb]
\centering
\begin{tabular}{| c | c | c | c |} 
 \hline
 Test Case & Values & Expected / Actual & IC\\ \hline
 keysThatMatchTest & "pattern" & Iterator (size 0) & I1, I2, I3 \\ \hline
\end{tabular}
\end{table}

\section{Edge Coverage}


\section{Prime Path Coverage}



\section{All-Uses Coverage }


\section{Logic-based Coverage}





\begin{thebibliography}{9}
\bibitem{silk} 
SiLK - CERT NetSA
\url{https://tools.netsa.cert.org/silk/docs.html}

\bibitem{ports} 
List of TCP and UDP port numbers
\url{https://en.wikipedia.org/wiki/List_of_TCP_and_UDP_port_numbers}



\end{thebibliography}


\bibliographystyle{plain}
\end{document}

%%% Local Variables:
%%% mode: latex
%%% TeX-master: t
%%% End:
